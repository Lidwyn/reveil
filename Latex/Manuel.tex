\documentclass{article}
\usepackage[T1]{fontenc} % Gestion des caractères (é et à)
\usepackage[utf8]{inputenc} % Encodage UTF-8
\usepackage{geometry} % Gestion des marges
\usepackage{fullpage} % Utiliser toute la page
\usepackage{enumitem} % listes personnalisées
\usepackage{amsmath} % Formules mathématiques
\usepackage{graphicx} % Insertion d'image
\usepackage[export]{adjustbox} % Ajuster la hauteur des images dans les minipages
\usepackage{caption} % Légendes
\usepackage{subcaption} % Sous légendes (pour plusieurs images en une)
\graphicspath{ {./fig/} } % Dossier d'images
\usepackage[many]{tcolorbox} % Boîtes colorées
\usepackage{indentfirst} % Identation du premier paragraphe
\usepackage{xparse} % Pour \NewDocumentCommand
\usepackage{fp}
\usepackage{tikz} % Graphique et annotations d'images
\usepackage{layout} % Affichage des marges
\usepackage{nameref} % Avoir le nom d'une section par un label
\usepackage{tocloft} % Pour gérer la map de la table des champ
\usepackage{bbding} % Avoir le caractère \Checkmark

\definecolor{boxlightgrey}{rgb}{0.8,0.8,0.8}
\definecolor{boxdarkgrey}{rgb}{0.25,0.25,0.25}

\geometry{ % zone de texte plus grande
    a4paper,
    hmargin = 3cm,
    top = 2cm
}

    % Packages mise en page : texte
\setlength\parindent{20pt}

    % Numero de section en lettre romaine
\renewcommand{\thesection}{\Roman{section}}
\renewcommand{\thesubsection}{\hspace{2pt} \Large \arabic{subsection}}
\renewcommand{\thesubsubsection}{\hspace{7pt} \large \arabic{subsection}.\arabic{subsubsection}}

    % Lettre encerclée
\newcommand*\circled[1]{\tikz[baseline=(char.base)]{
    \node[shape=circle,draw,inner sep=1pt] (char) {#1};}}
    % Annotation d'image

%\annotatedFigureBox{bottom left point}{top righ corner}{label}{label coordinate}[main color][backgroud color]
\NewDocumentCommand{\annotatedFigureBoxCustom}{m m m m O{black} O{white}}{
    \draw[#5,thick,rounded corners] (#1) rectangle (#2);
    \if\relax\detokenize{#3}\relax
        % empty
    \else
        \annotatedFigureLetter{#3}{#4}[#5][#6]
    \fi
}

%\annotatedFigureBox{xA}{yA}{width}{height}{label}{label pos}[main color][backgroud color]
\NewDocumentCommand{\annotatedFigureBox}{m m m m m m O{black} O{white}}{
    \FPeval{\xB}{#1 + #3}   % xB yB : top right corner
    \FPeval{\yB}{#2 + #4}
    \FPeval{\xC}{#1 + #3/2} % xC yC : midle point
    \FPeval{\yC}{#2 + #4/2}
    \ifnum#6=1 % getting label coordinate with the coordinate pos #6
        \def\a{#1}
        \def\b{#2}
    \else\ifnum#6=2
        \def\a{\xB}
        \def\b{#2}
    \else\ifnum#6=3
        \def\a{\xB}
        \def\b{\yB}
    \else\ifnum#6=4
        \def\a{#1}
        \def\b{\yB}
    \else\ifnum#6=5
        \def\a{\xC}
        \def\b{#2}
    \else\ifnum#6=6
        \def\a{\xB}
        \def\b{\yC}
    \else\ifnum#6=7
        \def\a{\xC}
        \def\b{\yB}
    \else\ifnum#6=8
        \def\a{#1}
        \def\b{\yC}
    \fi\fi\fi\fi\fi\fi\fi\fi
    \annotatedFigureBoxCustom{#1,#2}{\xB,\yB}{#5}{\a,\b}[#7][#8]
}

%\annotatedFigureLetterCustom{label}{x, y}[main color][backgroud color]
\NewDocumentCommand{\annotatedFigureLetter}{m m O{black} O{white}}{
    \node at (#2) [fill=#4,thick,shape=circle,draw=#3,inner sep=1pt,font=\sffamily,text=#3] {\small\textbf{#1}};
}

\newenvironment {annotatedFigure}[1]{\centering\begin{tikzpicture}[baseline=(top)]
    \node[anchor=south west,inner sep=0] (image) at (0,0) { #1};
    \begin{scope}[x={(image.south east)},y={(image.north west)}]}% BODY here
{\end{scope}
\path (current bounding box.north) ++(0pt,-\ht\strutbox) coordinate (top);
\end{tikzpicture}}

\title{MANUEL D'UTILISATION DU REVEIL}
\author{LIDWYN LE BARS}
\date{05 Juillet 2025}

\begin{document}
    % Ne pas afficher les \subsubsection{}
\addtocontents{toc}{\protect\setcounter{tocdepth}{2}}

\maketitle

\section{Introduction}

Ce réveil à pour fonctionnalité l'affichage de l'heure, et une alarme sonore qui se déclenche automatiquement à une heure réglé en amont par l'utilisateur.
En plus de ça, le réveil à la capacité d'enregistrées plusieurs réveils qui ont une heure de déclenchement et les jours de la semaine où ils se déclenchents.
On peut aussi inhibé l'affichage pour être dans le noir complet.

\section{Fonctionnalités et spécificités du réveil}

Le réveil permet l'affichage, le réglage d'heure et de la luminosité. Il permet de configurer plusieurs réveils en simultané, et les enregistres même quand il ne sont pas actifs.
Par exemple, un réveil à 8h les lundis, mardis et mercredis, et un autre à 8h15 les jeudis et vendredis. Les révéils peuvents aussi être à usage unique et s'auto désarmer après une utilisation.


\section{Élement du réveil}

Voici la liste des éléments sur le réveil:

\begin{figure}[h!]
    \centering
    \begin{annotatedFigure}
        {\includegraphics[width = .9\textwidth]{reveil.png}}
        \annotatedFigureBox{.026}{.08}{.608}{.833}{A}{5}[red]
        \annotatedFigureBox{.661}{.115}{.311}{.4}{B}{8}[red]
        \annotatedFigureBox{.678}{.53}{.19}{.4}{C}{4}[red]
        \annotatedFigureBox{.89}{.6}{.082}{.25}{D}{1}[red]
        \annotatedFigureBox{.651}{.065}{.321}{.05}{E}{8}[red]
        \annotatedFigureBox{.5}{1}{0}{0}{F}{1}[red]
    \end{annotatedFigure}
    \caption{Haching Tool}
    \label{fig:hachingTool1}
\end{figure}

\begin{itemize}
    \item \textbf{A} : Affichage 1 : heure : minute
    \item \textbf{B} : Affichage 2 : heure : minute de la prochaine alarme (éteind si null)
    \item \textbf{C} : 4 boutons de selections directionnels
    \item \textbf{D} : Bouton Setup
    \item \textbf{E} : 7 Leds pour l'affichage des jours de la semaine pour le réglage des alarmes
    \item \textbf{F} : un large bouton sur le dessus permettant l'arrêt de l'alarme ou le control de l'inibition de l'affichage
\end{itemize}

\section{Réglage de l'heure}

Pour le réglage de l'heure, il faut tout d'abord appuyer sur le bouton setup (bouton S).
Lorsqu'on appui sur le bouton setup, en entre en mode modification de l'heure. Les secondes sont affichées à droite et tout à part l'heure clignote.
L'élément qui ne clignote pas (par défaut : l'heure) est l'élement sélectionnée qui peut être modifié avec les flèches "plus" et "moins".
Pour passer d'un éléments à un autre, on utilise les flèches de gauche et de droite. Nous avons dans l'ordre:

Heures $\leftrightarrow$ Minutes $\leftrightarrow$ Secondes $\leftrightarrow$ Jours $\leftrightarrow$ Mois $\leftrightarrow$ Année $\leftrightarrow$ Jour de la semaine

Attention, ce réglage ne prend pas en comptes les années bissextile. Lors du passage de secondes à Jours,
L'inteface change vers la deuxième "page" où l'on retrouve le jours sur les deux premiers digits de l'affichage principal,
le mois sur les deux derniers digits de l'affichage principal, l'année sur l'affichage secondaire et le jour de la semaine (lundi, mardi etc.) est représenté sur les 7 leds en dessous du deuxième affichage, ce paramètre dois aussi être réglé à la main.

\begin{enumerate}
    \item appuyer sur le bouton menu, l'affichage clignote sauf l'heure, et l'affichage du prochain reveil passe à l'affichage des secondes.
    \item reglage de l'heure (et des autres paramètres à part les secondes) avec les boutons haut et bas.
    \item selections du paramètre via les touches gauche et droite dans l'ordre : Heures $\leftrightarrow$ Minutes $\leftrightarrow$ Secondes $\leftrightarrow$ Jours $\leftrightarrow$ Mois $\leftrightarrow$ Année $\leftrightarrow$ Jour de la semaine
    \item lorsque l'on clique sur le bouton droite alors que ce sont les secondes qui sont séléctionnés on passe de la première page à la deuxième et inversement si l'on clique sur le bouton gauche lorsque les jours sont séléctionnés pour passer de la deuxième page à la première.
\end{enumerate}

\subsection{Appuyer sur le bouton menu}

Pour commencer on appui sur le bouton menu/setup \circled{\textbf{A}}. 
On passe en mode réglage, et l'affichage de l'heure \circled{\textbf{B}} ne clignote pas tandis que l'affichage de minutes \circled{\textbf{C}} et des secondes \circled{\textbf{D}} eux clignotent.

\begin{figure}[h!]
    \centering
    \begin{annotatedFigure}
        {\includegraphics[width = .9\textwidth]{reveil.png}}
        \annotatedFigureBox{.026}{.08}{.288}{.833}{B}{2}[red]
        \annotatedFigureBox{.342}{.08}{.288}{.833}{C}{2}[red]
        \annotatedFigureBox{.661}{.115}{.311}{.4}{D}{8}[red]
        \annotatedFigureBox{.89}{.6}{.082}{.25}{A}{1}[red]
    \end{annotatedFigure}
    \caption{passage en mode modification}
    \label{bouton menu}
\end{figure}

\clearpage  %______________________________/!\_clearpage_/!\_____________________________

\subsection{Réglage des paramètres}

Dans ce mode on peut régler les paramètres lorsqu'ils sont sélectionnés. Pour se faire, il faut utiliser les bouton plus et moins \circled{\textbf{D}}.

\begin{figure}[h!]
    \centering
    \begin{annotatedFigure}
        {\includegraphics[width = .9\textwidth]{reveil.png}}
        \annotatedFigureBox{.738}{.53}{.07}{.4}{D}{4}[red]
    \end{annotatedFigure}
    \caption{bouton haut et bas}
    \label{bouton haut et bas}
\end{figure}

\subsection{Séléction des paramètres}

Il y a 7 paramètres réglables dans l'ordre :
\begin{itemize}
    \item Heure \circled{\textbf{G}}
    \item Minute \circled{\textbf{H}}
    \item Seconde \circled{\textbf{I}}
    \item Jour \circled{\textbf{J}}
    \item Mois \circled{\textbf{K}}
    \item Année \circled{\textbf{L}}
    \item Jour de la semaine \circled{\textbf{M}}
\end{itemize}

Pour passer d'un paramètre à l'autre, il faut utiliser les boutons droite \circled{\textbf{E}} et gauche \circled{\textbf{F}}.

\begin{figure}[h!]
    \centering
    \begin{annotatedFigure}
        {\includegraphics[width = .9\textwidth]{reveil.png}}
        \annotatedFigureBox{.675}{.6}{.082}{.25}{E}{1}[red]
        \annotatedFigureBox{.79}{.6}{.082}{.25}{F}{2}[red]
        \annotatedFigureBox{.026}{.08}{.288}{.833}{G}{3}[red]
        \annotatedFigureBox{.344}{.08}{.288}{.833}{H}{1}[red]
        \annotatedFigureBox{.661}{.115}{.15}{.4}{I}{8}[red]
    \end{annotatedFigure}
    \caption{bouton droite et gauche, heure, minute et seconde}
    \label{affichage 1 reglage}
\end{figure}

\clearpage  %______________________________/!\_clearpage_/!\_____________________________

Lorsque les secondes sont sélectionnées et qu'on appui sur le bouton droite, on passe sur le deuxième page.
Pour revenir au premier page, il faut appuyer sur le bouton gauche lorsque les jours sont séléctionnés.

\begin{figure}[h!]
    \centering
    \begin{annotatedFigure}
        {\includegraphics[width = .9\textwidth]{reveil.png}}
        \annotatedFigureBox{.026}{.08}{.288}{.833}{J}{3}[red]
        \annotatedFigureBox{.344}{.08}{.288}{.833}{K}{1}[red]
        \annotatedFigureBox{.661}{.115}{.311}{.4}{L}{8}[red]
        \annotatedFigureBox{.651}{.065}{.321}{.05}{M}{8}[red]
    \end{annotatedFigure}
    \caption{Affichage 2}
    \label{affichage 2 reglage}
\end{figure}

\clearpage  %______________________________/!\_clearpage_/!\_____________________________

\section{Selections d'une alarme}

Les alarmes ont un menu associés qui permet :

\begin{itemize}
    \item L'activation ou la désactivation des alarmes
    \item La création de nouvelles alarmes
    \item La modification de l'alarmes existantes
\end{itemize}

Ce menu est accèssible en appuyant sur le bouton droite.

\subsection{fonctionnement des alarmes enregistrées}

\subsection{Création d'une nouvelle alarme}

\subsection{Modification d'une alarme existante}
\end{document}